

\chapter{objetivo general}

El objetivo general del proyecto es que se estudie, por medio de simulación de Monte Carlos el modelo teórico "dark Susy", mediante la obtención teórica de las propiedades del fotón oscuro en un entorno simulado de los detectores que realizarían esta detección en dos configuraciones ya conocidas del experimento CMS, en la llamada fase-2 y en la de alta luminosidad.

%El objetivo general del proyecto el estudio de un modelo bien fundamentado de produccion de particulas de materia oscura. Se pretende la implementacion de dicho modelo es paquetes de simulacion propios del area de altas energias.  Caracterizacion de los parametros de dicho modelo, lo cual incluye propiedades del foton oscuro como lo son, el tiempo de vida, masa invariante, modos decaimiento, entre otros.  Adicionalmente se pretende el analisis de la respuesta del detector a dichas senales hipoteticas, tratando de extraer eficiencias de deteccion, resoluciones y metodos de supresion de ruido.  Finalmente se busca analizar las posibilidaddes de observacion de dicha particula usando la configuracion actual del detector CMS del CERN y la configuracion futura en la llamada fase-2 o de alta luminosidad. 


