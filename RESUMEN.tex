\newcommand{\HRule}{\rule{\linewidth}{0.5mm}} 

\begin{titlepage}
\center
%	HEADING & LOGO
\textsc{
\Huge{Universidad de Sonora}\\[.5cm]
\Large
Departamento de Investigaci\'on en F\'{\i}sica\\[1cm] 
\includegraphics[width=3cm]{unison}\\[3cm]
Anteproyecto de Tesis de Maestria (version 2.0). \\[.7cm] 
Proponente: Francisco Mart\'{\i}nez S\'anchez}\\[.7cm] 

%	TITLE 
\sffamily
\HRule \\[0.4cm]
\textbf{\LARGE Estudios de simulación en la búsqueda de nuevos
bosones ligeros durante la fase de alta luminosidad del
experimento CMS del CERN}\\[0.2cm] 
\HRule \\[3cm]
 
%	AUTHORS & SUPERVISOR
\large
\begin{minipage}[t]{.6\textwidth}
\begin{flushleft}
Miembros del comit\'e:
\\
Alfredo Mart\'{\i}n Casta\~neda Hernandez (Director)\\
Susana Alvarez Garcia (tutor)\\
Marcelino Barbosa Flores (tutor)\\
\end{flushleft}

\end{minipage}\hfill
%\begin{minipage}[t]{.4\textwidth}

%\begin{flushright}
%\emph{Course Instructor} 

%Sandipan Bandyopadhyay\\ 

%\end{flushright}
%\end{minipage}
%\\[2cm]

%	DATE
{\today}\\[3cm]

\end{titlepage}


\begin{abstract}
El Modelo Estándar es la teoría cuántica de campo que en la actualidad describe de manera mas precisa las interacciones entre las partículas fundamentales y los diferentes tipos de fuerzas que experimentan las mismas. En el 2012 las colaboraciones experimentales Aparato Toroidal (ATLAS) y Solenoide Compacto de Muones (CMS) del CERN reportaron el descubrimiento de una nueva partícula cuyas propiedades son consistentes con el bosón de Higgs. Dicha particulas esta asociada con el campo de Higgs, mecanismo por el cual las partículas elementales adquieren masa. Esto significa un triunfo para el modelo estandar, sin embargo aun existen varios fenómenos que dicho modelo no puede dar explicacion como lo es la materia oscura de la que se desconoce su composición y cuya existencia se infiere por los resultados de su interacción con la materia visible que se encuentra a su alrededor. El presente proyecto explora un modelo teorico que predice la creación de partículas de materia oscura producto de las colisiones de protones que viajan a velocidades relativistas como las producidas en el Gran Colisionador de Hadrones del CERN. Este modelo es estudiado por medio de simulación de Monte Carlo donde se explora sus diferentes propiedades y se simula la respuesta del detector al paso de esta particula, todo esto en el contexto del experimento CMS del CERN.  
\end{abstract}
