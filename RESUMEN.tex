\newcommand{\HRule}{\rule{\linewidth}{0.5mm}} 

\begin{titlepage}
\center
%	HEADING & LOGO
\textsc{
\Huge{Universidad de Sonora}\\[.5cm]
\Large
Departamento de Investigaci\'on en F\'{\i}sica\\[1cm] 
\includegraphics[width=3cm]{unison}\\[3cm]
Anteproyecto de Tesis de Maestria \\[.7cm] 
Proponente: Francisco Mart\'{\i}nez S\'anchez}\\[.7cm] 

%	TITLE 
\sffamily
\HRule \\[0.4cm]
\textbf{\LARGE Estudios de simulación en la búsqueda de nuevos
bosones ligeros durante la fase de alta luminosidad del
experimento CMS del CERN}\\[0.2cm] 
\HRule \\[3cm]
 
%	AUTHORS & SUPERVISOR
\large
\begin{minipage}[t]{.6\textwidth}
\begin{flushleft}
Miembros del comit\'e:
\\
Alfredo Mart\'{\i}n Casta\~neda Hernandez (Director)\\
Susana Alvarez Garcia (tutor)\\
Marcelino Barbosa Flores (tutor)\\
\end{flushleft}

\end{minipage}\hfill
%\begin{minipage}[t]{.4\textwidth}

%\begin{flushright}
%\emph{Course Instructor} 

%Sandipan Bandyopadhyay\\ 

%\end{flushright}
%\end{minipage}
%\\[2cm]

%	DATE
{\today}\\[3cm]

\end{titlepage}


\begin{abstract}
El modelo estándar es la teoría cuántica de campo que en la actualidad describe de manera precisa las interacciones entre las partículas fundamentales y los diferentes tipos de fuerzas que experimentan las mismas. En el 2012 las colaboraciones experimentales Aparato Toroidal del LHC (ATLAS) y Compact Muon Solenoid (CMS) reportaron el descubrimiento de una nueva partícula cuyas propiedades son consistentes con el bosón de Higgs, portador del llamado campo de Higgs, mecanismo por el cual las partículas adquieren masa. Actualmente las propiedades de estas partículas son estudiadas desde el modelo estándar, sin embargo el mismo falla en dar explicación a varios fenómenos relacionados con la materia oscura de la que se desconoce su composición y cuya existencia se infiere por los resultados de su interacción con la materia visible que se encuentra a su alrededor.

El presente proyecto explora varios modelos teóricos que predicen la creación de partículas de materia oscura producto de las colisiones de protones que viajan a velocidades relativistas como las producidas en el Gran Colisionador de Hadrones del CERN. Estos modelos son estudiados por medio de simulación de Monte Carlo donde se explora sus diferentes propiedades y se simula la respuesta del detector, en el contexto del experimento CMS del CERN.  
\end{abstract}
