\chapter{Ojetivos específicos}

Los siguientes objetivos específicos a alcanzar los siguientes:
\begin{itemize}
\item Caracterizar el modelo "dark SUSY" por medio de su implementacion en paquetes de simulacion de altas energia, lo anterior requiere del manejo de paquetes propios de altas energias como lo son FeynRules, Madgraph, Pythia y Delphes.
\item Extraccion de propiedades del modelo como lo son las propiedades basicas del foton oscuro, masa, tiempo de vida, modos de decaimientos. Lo anterior para verificar las
  predicciones del modelo teorico.
\item Estudio de la parte experimental, desarrollo de codigos de analisis que permitan extraer la eficiencia de deteccion de las particulas producidas del decaimiento del foton oscuro, el trabajo se centrara en estudiar el decamiento a pares de muones de signo opuesta, exploracion de propiedades como funcion de tiempo de vida del foton oscuro
\item Optimizacion de la seleccion de eventos usando como referencia las caracteristicas del modelo, es decir implementar cortes en umbrales de energia, masa, tiempo de vida propios del foton oscuro buscando suprimir la contribucion de procesos de ruido del modelo estandar 
\item Comparar los resultados obtenidos usando diferentes configuraciones del detector CMS, en particular comparando la actual configuracion (usada hasta 2018) y la future configuracion (fase de alta luminosidad) buscando caracterizar por medio de simulacion la posible mejora en la senal con las futuras acutalizaciones del experimento CMS. 
\end{itemize}
