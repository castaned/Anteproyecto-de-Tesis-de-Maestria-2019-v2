\chapter{Ojetivos específicos}

Los siguientes objetivos específicos a alcanzar los siguientes:
\begin{itemize}
\item Caracterizar el modelo "dark SUSY" por medio de su implementación en paquetes de simulación de altas energía, lo anterior requiere del manejo de paquetes propios de altas energías como lo son FeynRules, Madgraph, Pythia y Delphes.
\item Extracción de propiedades del modelo como lo son las propiedades básicas del fotón oscuro, masa, tiempo de vida, modos de decaimientos. %Lo anterior para verificar las  predicciones del modelo teórico.
\item Desarrollo de códigos de análisis que permitan extraer la eficiencia de detección de las partículas producidas del decaimiento del fotón oscuro, el trabajo se centrara en estudiar el decaimiento a pares de muones de signo opuesta, exploración de propiedades como función de tiempo de vida del fotón oscuro y optimización de la selección de eventos.
%\item Optimización de la selección de eventos usando como referencia las características del modelo, implementando cortes en umbrales de energía, masa, tiempo de vida propios del fotón oscuro buscando suprimir la contribución de procesos de ruido del modelo estándar. 
\item Comparación de los resultados obtenidos usando dos diferentes configuraciones del detector CMS, la actual configuración (usada hasta 2018) y la futura configuración (fase de alta luminosidad).% buscando caracterizar por medio de simulación la posible mejora en la señal con las futuras acutalizaciones del experimento CMS. 

%%%%%%%%%%% ORIGINAL CONTENIDO %%%%%%%%%%%%
%\item Caracterizar el modelo "dark SUSY" por medio de su implementación en paquetes de simulación de altas energía, lo anterior requiere del manejo de paquetes propios de altas energías como lo son FeynRules, Madgraph, Pythia y Delphes.
%\item Extracción de propiedades del modelo como lo son las propiedades básicas del fotón oscuro, masa, tiempo de vida, modos de decaimientos. Lo anterior para verificar las  predicciones del modelo teórico.
%\item Estudio de la parte experimental, desarrollo de códigos de analisis que permitan extraer la eficiencia de detección de las particulas producidas del decaimiento del foton oscuro, el trabajo se centrara en estudiar el decamiento a pares de muones de signo opuesta, exploracion de propiedades como funcion de tiempo de vida del foton oscuro
%\item Optimizacion de la seleccion de eventos usando como referencia las caracteristicas del modelo, es decir implementar cortes en umbrales de energia, masa, tiempo de vida propios del foton oscuro buscando suprimir la contribucion de procesos de ruido del modelo estandar 
%\item Comparar los resultados obtenidos usando diferentes configuraciones del detector CMS, en particular comparando la actual configuracion (usada hasta 2018) y la future configuracion (fase de alta luminosidad) buscando caracterizar por medio de simulacion la posible mejora en la senal con las futuras acutalizaciones del experimento CMS. 
\end{itemize}
