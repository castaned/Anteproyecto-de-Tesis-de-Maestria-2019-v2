\chapter{Propuesta}

El presente trabajo se propone hacer un estudio, por medio de simulación de Monte Carlo, de los modelos teóricos que predicen la creación de nuevas partículas como producto de la colisión de protones altamente energéticos. Estas nuevas partículas serían candidatas a explicar la composición de la materia oscura. Usualmente en dichos modelos las nuevas partículas interaccionan débilmente con el sector del modelo estándar, es decir con la materia visible, por lo que su detección se dará de forma indirecta, o en otras palabras, por su decaimiento a partículas conocidas del modelo estándar []. Adicionalmente al estudio de los modelos teóricos se pretende trabajar en la parte experimental, la cual consiste en el estudio de la respuesta del detector al paso de las partículas elementales y la extracción de los observables experimentales como son la energía de las partículas, el momento y la trayectoria, entre otros. La parte experimental es fundamental ya que sin una buena estrategia de selección de datos, técnicas de supresión de ruido y optimización de la señal sería imposible la observación de esta nueva física. En este proyecto se considera el detector CMS del CERN como el aparato experimental que proporcionará los datos de estudio, ya sea por simulación o por uso de datos reales.


En una primera fase se pretende analizar los modelos teóricos de una manera fenomenológica, es decir por medio de paquetes de simulación propios del área de altas energías, además de entender la respuesta del detector a las nuevas señales buscando optimizar la selección de los eventos en base a las propiedades de cada modelo. En una segunda fase se pretende estudiar la señal de dichos modelos bajo diferentes escenarios del detector CMS, un primer escenario sería la configuración actual del detector CMS, que es la configuración usada hasta el 2018, durante el llamado período 'Run-2' y comparar los resultados con el detector que se tiene propuesto para la fase de alta luminosidad o también llamada Phase-2, la cual empezará a partir del 2025; de esta manera se puede predecir en base a los estudios de simulación las posibilidades de identificación de una nueva señal en los próximos años y cómo la actualización de los detectores y métodos de identificación de
partículas podrían contribuir a incrementar las probabilidades de descubrimiento de estas nuevas partículas.

