\chapter{Metodología}

Con el fin de alcanzar los objetivos planteados se plantea la siguiente secuencia de actividades:


\begin{itemize}
    \item Producción de muestras de Monte Carlo (2 meses): Se espera implementar el modelo "dark SUSY" usando los paquetes propios del area y producir muestras de simulación para el proceso de senal. Las muestras se producirán usando los recursos computacionales de la Universidad de Sonora. Este paso requiere del desarrollo de código para la distribución de las corridas de simulación en forma paralela usando el cluster ACARUS. Los paquetes de simulación que se usarán serán FeynRules, MADGRAPH [], Pythia y Delphes [].
    \item Análisis preliminar (2 meses): El estudiante debe desarrollar diferentes herramientas de análisis de datos con el fin de acceder a los datos producidos en la simulación y extraer las variables de interés como lo son las propiedades del foton oscura (masa, tiempo de vida, etc.). Comúnmente dichas herramientas de análisis consisten en códigos escritos en lenguaje C++ y python, de esta manera el estudiante desarrollará habilidad en la manipulación de muestras de datos.
    \item Optimización de la selección de eventos (2 meses): Después del acceso de datos de simulación y variables de interés se procederá al estudio de la selección de eventos, la cual a grandes rasgos consiste en seleccionar el conjunto de variables físicas y valores que puedan optimizar el proceso de señal y reducir lo más posible la contribución del ruido.
    \item Análisis estadístico (3 meses): Se realizará un estudio estadístico en el cual se extraerán el número de eventos de señal y ruido después de la selección. De ahí se puede interpretar los resultados en base a indicadores estadísticos y concluir la probabilidad de observación con datos recolectados en los próximos años.
    \item Comparacion de los resultados obtenidos usando la configuracion actual del detector CMS y la configuracion que se utilizara en la llamada fase-2 o de alta luminosidad. 
    \item Escritura de Tesis (3 meses): Se presentará los resultados a expertos del área buscando una retroalimentación, procediéndose al mismo a la escritura del trabajo de tesis.
\end{itemize}
